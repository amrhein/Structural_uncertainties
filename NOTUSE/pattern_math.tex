%% LyX 2.2.0 created this file.  For more info, see http://www.lyx.org/.
%% Do not edit unless you really know what you are doing.
\documentclass[12pt,english]{article}
\usepackage{mathptmx}
\renewcommand{\familydefault}{\rmdefault}
\usepackage[T1]{fontenc}
\usepackage[latin9]{inputenc}
\usepackage{geometry}
\geometry{verbose,tmargin=1in,bmargin=1in,lmargin=1in,rmargin=1in}
\usepackage{amstext}
\usepackage{setspace}
\usepackage[authoryear]{natbib}
\usepackage{babel}
\begin{document}

\title{Computing effective degrees of freedom ($N_{ef}^{\star}$) as a function
of frequency}

\maketitle
\begin{singlespace}
Bretherton et al. (1998) gives
\[
N_{ef}^{\star}=\frac{\left(\sum\lambda_{i}\right)^{2}}{\left(\sum\lambda_{i}^{2}\right)}=\frac{\text{tr}\left(\mathbf{C}\right)^{2}}{\text{tr}\left(\mathbf{C}^{2}\right)}
\]
where $\lambda_{i}$ are the eigenvalues of the covariance matrix
\[
\mathbf{C}=\frac{1}{n-1}\mathbf{X}\mathbf{X}^{\top}
\]
where $n$ is the number of times and $\mathbf{X}$ is the data matrix.
The $\lambda_{i}$ are also the squared singular values of the matrix
of model output $\mathbf{X}$, so that 
\[
\mathbf{X}=\mathbf{U}\Lambda^{\frac{1}{2}}\mathbf{V}^{\top}
\]
where $\mathbf{\Lambda}$ is populated along its diagonal with the
$\lambda_{i}$. Our goal is to compute a function 
\[
N_{ef}^{\star}\left(\nu\right)
\]
that is a function of frequency, i.e. what are the degrees of freedom
as a function of time scale? 

The component of $\mathbf{X}$ that varies at a single frequency $\nu$,
$\mathbf{X}_{\nu}$, can be isolated by considering only the components
of the principal components (the right singular vectors in the columns
of $\mathbf{V}$) at that frequency,
\[
\mathbf{X}_{\nu}=\mathbf{U}\Lambda^{\frac{1}{2}}\mathbf{V}_{\nu}^{\top}.
\]
Our goal is to compute $N_{ef}^{\star}\left(\nu\right)$ using the
trace form of the definition. The trace of a covariance matrix is
equal to the sum of squared diagonal elements of any diagonalizing
basis. Here we choose the EOF basis for the full $\mathbf{X}$; we
will project $\mathbf{X}$ at varying frequencies onto this basis
and compute the squared weights to get the trace. The squared projection
of $\mathbf{X}_{\nu}$ is then 
\[
\lambda_{\nu\,i}=\lambda_{i}\left|\hat{\mathbf{v}}_{i}\left(\nu\right)\right|^{2}
\]
where the latter term ($\left|\hat{\mathbf{v}}_{i}\left(\nu\right)\right|^{2}$)
can be obtained via a power spectral density estimate. Note that the
eigenvectors of $\mathbf{C}^{2}$ are the same as for $\mathbf{C}$
and the eigenvalues are the square. Thus we arrive at

\[
N_{ef}^{\star}\left(\nu\right)=\frac{\left(\sum\lambda_{i}\left|\hat{\mathbf{v}}_{i}\left(\nu\right)\right|^{2}\right)^{2}}{\sum\lambda_{i}^{2}\left|\hat{\mathbf{v}}_{i}\left(\nu\right)\right|^{4}}.
\]
For a simple example, consider the case where the field is 2x1 with
a global mode with frequency .1 and a top-only mode with frequency
.01... 
\end{singlespace}

\end{document}
